% ====== TAREA 2 DE PROBABILIDAD ======

\documentclass[12pt,a4paper]{report}
\usepackage[utf8x]{inputenc}
\usepackage{amsmath}
\usepackage{amsfonts}
\usepackage{amssymb}
\usepackage{graphicx}
\usepackage{enumitem}
\usepackage{calrsfs}

\newcommand*{\Comb}[2]{{}^{#1}C_{#2}}

\begin{document}
\begin{titlepage}
	\centering
	{\scshape\LARGE Universidad Nacional Autónoma de México \par}
	\vspace{1cm}
	{\scshape\Large Probabilidad I\par}
	\vspace{1.5cm}
	{\huge\bfseries Tarea II\par}
	\vspace{.5cm}
	{\Large\itshape Sandra Del Mar Soto Corderi \par}
	\vspace{.5cm}
	{\Large\itshape Edgar Quiroz Castañeda \par}
    \vspace{.5cm}
	{\Large\itshape Raúl Llamosas Alvarado \par}
	 \vspace{.5cm}
	{\Large\itshape Alan Ernesto Arteaga Vázquez \par}
	\vfill
	 \includegraphics[width=0.5\textwidth]{escudo.png}
	\vfill

% Bottom of the page
	{\large Martes 4 de septiembre del 2018 \par}
\end{titlepage}

\pagebreak
\setlength{\voffset}{-0.75in}
\setlength{\headsep}{5pt}

\begin{enumerate}
   % Ejercicio 1
   \item {
  La diferencia simétrica de dos conjuntos A,B se define como:\\

		$$A\triangle B = (A \cap B^c) \cup (B \cap A^c)=(A \cup B) \setminus (B \cap A)$$
Demuestre
\begin{enumerate}[label=\alph*) ]
	%a
	\item{
		$A^c \triangle B^c = A \triangle B $ \\

		Notemos que tanto la unión como la itersección conmutan, y que el complemento
		del complemento es el conjunto original.\\
		De estas tres cosas y de la definición de diferencia simétrica, tenemos que
		\begin{align*}
			A^c \triangle B^c &= (A^c \cap (B^c)^c)) \cup (B^c \cap (A^c)^c))\\
												&= (A^c \cap B) \cup (B^c \cap A)\\
												&= (B \cap A^c) \cup (A \cap B^c)\\
												&= (A \cap B^c) \cup (B \cap A^c)\\
												&= A \triangle B
		\end{align*}
	}

	%b
	\item{
		$ (A_{1} \cup A_{2})\triangle (B_{1} \cup B_{2})\subseteq (A_{1} \triangle B_{1}) \cup (A_{2} \triangle B_{2}) $ \\

		Sea $x \in (A_{1} \cup A_{2})\triangle (B_{1} \cup B_{2})
		= ((A_{1} \cup A_{2}) \cap (B_{1} \cup B_{2})^c) \cup ((B_{1} \cup B_{2}) \cap (A_{1} \cup A_{2})^c)$.\\
		Esto es, que $x \in (A_{1} \cup A_{2})$ o $x \in (B_{1} \cup B_{2})$,
		pero no los dos al mismo tiempo.\\
		Primero veamos que pasa si $x \in (A_{1} \cup A_{2})$.\\
		Entonces $x \notin (B_{1} \cup B_{2})$ y $x \in A_1$ o  $x \in A_2$.
		Sin perdidad de generalidad, digamos que $x \in A_1$. Entonces en particular
		$x \in (A_1 \cup B_1)$. Además, como $x \notin (B_{1} \cup B_{2})$,
		entonces tenemos que $x \notin B_1$ ni $x \notin B_2$, por lo que en particular
		$x \notin (B_1 \cap A_1)$. Entonces $x \in ((A_1 \cup B_1) \setminus (B_1 \cap A_1)) =
		(A_{1} \triangle B_{1})$.\\
		Analogamente, si $x \in A_2$, tenemos que $x \in (A_2 \triangle B_2)$.\\
		Luego, si $x \in (B_{1} \cup B_{2})$, entonces $x \notin (A_{1} \cup A_{2})$ y
		$x \in B_1$ o  $x \in B_2$.\\
		Como en el caso anterior, si $x \in B_1$, entonces $x \in (B_1 \cup A_1)$ y
		$x \notin (A_1 \cap B_1)$, por lo que $x \in A_{1} \triangle B_{1}$.\\
		Y analogamente, si $x \in B_2$, pasa que $x \in A_2 \triangle B_2$.\\
		Entonces, sin importar el elemento que tomemos de $(A_{1} \cup A_{2})\triangle (B_{1} \cup B_{2})$,
		éste está contenido en $(A_{1} \triangle B_{1}) \cup (A_{2} \triangle B_{2})$.\\
	}

	%c
	\item{
		Sea $\Omega$ un conjunto y $\mathcal{F} \subseteq P(\Omega)$ una $\sigma$-algebra.
		Demuestre que $A \triangle B \in \mathcal{F} $ siempre que $A,B \in \mathcal{F}$. \\\\
		Como $\mathcal{F}$ es una $\sigma$-algebra, entonces es cerrada bajo el complemento y la unión. Entonces
		\begin{align*}
			A,B \in \mathcal{F} &\implies A^c,B^c \in \mathcal{F}\\
													&\implies (A^c \cup B),(B^c \cup A) \in \mathcal{F}\\
													&\implies (A^c \cup B)^c = (A \cap B^c),(B^c \cup A)^c = (B \cap A^c) \in \mathcal{F}\\
													&\implies (A \cap B^c) \cup (B \cap A^c) = A \triangle B \in \mathcal{F}.
		\end{align*}

	}



\end{enumerate}

	}

	% Ejercicio 2
   \item {
    Si $\Omega = \lbrace 1,2,3,4,5 \rbrace$ construya una $\sigma$-álgebra $\mathcal{F} \neq \mathcal{P}(\Omega)$ tal que:\\
    $$\lbrace \lbrace 1,2 \rbrace , \lbrace 1,3 \rbrace \rbrace \subset \mathcal{F}$$

		Se puede construir una $\sigma$-álgebra agregando los elementos faltantes
		para que cumpla los axiomas.\\
		Primero, toda $\sigma$-algebra contiene al vacío y al universo, por lo que
		$\Omega, \emptyset \in \mathcal{F}$.\\
		Faltan además los complementos de los elementos dados, que son $\lbrace 3, 4, 5 \rbrace, \\
		\lbrace 2, 4, 5 \rbrace \in \mathcal{F}$.\\
		Y faltan las uniones de los elementos actuales, que son  $\lbrace 1, 2, 3 \rbrace,
		\lbrace 1, 2, 4, 5 \rbrace, \lbrace 1, 4, 5 \rbrace, \\
		\lbrace 2, 3, 4, 5 \rbrace \in \mathcal{F}$.\\
		Entonces faltarían los complementos de los nuevos eventos, que son
		$\lbrace 4, 5 \rbrace, \lbrace 3 \rbrace, \lbrace 2, 3 \rbrace, \lbrace 1 \rbrace
		\in \mathcal{F}$.\\
		Y nuevamente faltarían las nuevas posibles uniones, que solo es una en este caso.
		$\lbrace 1, 3, 4, 5 \rbrace \in \mathcal{F}$\\
		Y al llegar a este punto, nuestro conjunto ya es cerrado baja la unión y el
		complemento.
		Entonces una $\sigma$-álgebra que cumple lo requierido es\\
		$\mathcal{F} = \lbrace \emptyset,
		\lbrace 3 \rbrace, \lbrace 1 \rbrace,
		\lbrace 4, 5 \rbrace, \lbrace 2, 3 \rbrace, \lbrace 1,2 \rbrace , \lbrace 1,3 \rbrace,
		\lbrace 1, 2, 3 \rbrace, \lbrace 3, 4, 5 \rbrace, \lbrace 2, 4, 5 \rbrace, \lbrace 1, 4, 5 \rbrace, \\
		\lbrace 1, 2, 4, 5 \rbrace, \lbrace 2, 3, 4, 5 \rbrace,
		\Omega \rbrace$
	}


	% Ejercicio 3
   \item {
   En una universidad se ofrecen tres cursos de idiomas: uno  de inglés, uno de francés y otro de alemán.
	 Cualquiera de los 100 alumnos de la escuela puede tomar dichos cursos.
	 Hay 26 estudiantes en la clase de inglés, 28 en la de francés y 16 en la de alemán.
	 Hay 12 estudiantes que están en el de inglés y francés, 4 en el de ingles y alemán ,
	 6 en el de alemán y francés y 2 tomando los tres cursos\\

	 Digamos que I representa a los estudiantes de inglés, F de francés y A de alemán.
	 Entonces las cantidades de alumnos se puede representar como probabilidades.

		\begin{center}
		 \begin{tabular}{|c|c|}
							\hline
							Idioma & Porcentaje  \\
							\hline
							I & 26\% \\
							\hline
							F & 28\% \\
							\hline
							A & 16\% \\
							\hline
							I y F & 12\% \\
							\hline
							I y A & 4\% \\
							\hline
							F y A & 6\% \\
							\hline
							I, F y A & 2\% \\
							\hline
			\end{tabular}
		\end{center}

	\begin{enumerate}[label=\alph*) ]
	% a)
   \item {
		Si un estudiante se selecciona aleatoriamente ¿Cual es la probabilidad de que no esté inscrito en ningún curso de idiomas?\\

		El evento que corresponde a todos los alumnos estudiando idiomas es
		$I \cup F \cup A$, entonces lo contrario, el evento que corresponde a los
		alumnos que no toman ningún idioma, es $(I \cup F \cup A)^c$.
		Entonces
		\begin{align*}
			P((I \cup F \cup A)^c) &= 1 - P(I \cup F \cup A)\\
														 &= 1 - (P(I) + P(F) + P(A) - P(I \cap F))\\
														 &- P(I \cap A) - P(F \cap A) + P(I \cap F \cap A))\\
														 &= 1 - (0.26 + 0.28 + 0.16 - 0.12 - 0.04 - 0.06 + 0.02)\\
														 &= 1 - 0.5\\
														 &= 0.5
		\end{align*}
		Entonces hay una probabilidad de 0.5 de que el alumno elegido no esté tomando ningún idioma.
   }

   % b)
   \item {
   Si un estudiante se selecciona aleatoriamente, ¿Cual es la probabilidad de que lleve exactamente uno de los cursos de idiomas?\\

	 Consideremos primero a los alumnos que sólo estudian inglés.
	 Este evento es $I' = I \setminus (F \cup A)$.
	 Entonces
	 \begin{align*}
	 	P(I') &= P(I \setminus (F \cup A))\\
					&= P(I) - P(I \cap (F \cup A))\\
					&= 0.26 - P((I \cap F) \cup (I \cap A))\\
					&= 0.26 - (P(I \cap F) + P(I \cap A) - P(I \cap A \cap F))\\
					&= 0.26 - (0.12 + 0.04 - 0.02)\\
					&= 0.26 - 0.14\\
					&= 0.12
	 \end{align*}
   }

	 Análogamente, los eventos que corresponden a los alumnos que sólo estudian
	 alemán y francés son $F' = F \setminus (I \cup A)$  y $A' = A \setminus (F \cup I)$.\\
	 Y sus probabilidades son
	 \begin{align*}
	 	P(F') &= P(F) - (P(F \cap I) + P(F \cap A) - P(I \cap A \cap F))\\
					&= 0.28 - (0.12 + 0.06 - 0.02)\\
					&= 0.28 - 0.16\\
					&= 0.12
	 \end{align*}
	 Y
	 \begin{align*}
	 	P(A') &= P(A) - (P(A \cap F) + P(A \cap I) - P(I \cap A \cap F))\\
					&= 0.16 - (0.06 + 0.04 - 0.02)\\
					&= 0.16 - 0.08\\
					&= 0.08
	 \end{align*}

	 Entonces el evento que corresponde a que un estudiante sólo esté inscrito en
	 un idioma es $P(I' \cup F' \cup A')$.\\
	 Estos eventos son ajenos, pues precisamente corresponden a personas
	 estudiando únicamente un idioma, no más.\\
	 Entonces la probabilidad de este evento es
	 $$P(I' \cup F' \cup A') = P(I') + P(F') + P(A') = 0.12 + 0.12 + 0.8 = 0.32$$
	 Entonces hay una probabilidad de 0.32 de que un estudiante seleccionado
	 aleatoriamente estudie sólo un idioma.

    % c)
   \item {
  	Si dos estudiantes se seleccionan aleatoriamente ¿Cual es la probabilidad de que al menos uno esté tomando alguna clase de idiomas?\\

		Llamemos a este evento $A$.\\
		Consideremos el evento contrario, es decir que escogiendo dos personas
		aleatoriamente, ninguna esté estudiando un idioma.\\
		En el inciso anterior se calculó la probabilidad de no estar estudiando ningún idioma como 0.5.\\
		Teniendo en cuenta que hay 100 estudiantes, esto significa
		50 alumnos que no estudian idiomas.\\
		Entonces la probabilidad de elegir a dos estudiante que no lleven idiomas es la cantidad de maneras de elegir dos estudiantes del conjunto de estudiantes que no llevan idioma entre la cantidad
		total de maneras de elegir dos estudiantes.\\
		Esto es
 		\begin{equation*}
		P(A^c)  = \frac{\binom{50}{2}}{\binom{100}{2}}
                = \frac{(50)(49)}{(100)(99)}
                = \frac{49}{(2)(99)}
                = \frac{49}{198}
		\end{equation*}
		Entonces la probabilida de $A$ es

        \begin{equation*}
        P(A)    = 1 - P(A^c)
                = 1 - \frac{49}{198}
                = \frac{198 - 49}{198}
                = \frac{149}{198}
                \approx 0.752
        \end{equation*}



   }


	\end{enumerate}

	}

	% Ejercicio 4
   \item {
  	Si $P(A) = \frac{1}{3}$ y $P(B^c)=\frac{1}{4}$. ¿Pueden ser A y B mutuamente excluyentes?\\

		Notemos que como $P(B^c)=\frac{1}{4}$, entonces $P(B)=1 - P(B^c) = 1 - \frac{1}{4} = \frac{3}{4}$.\\
		Luego, supongamos que A y B son mutuamente excluyentes. Por lo tanto
		$P(A \cup B) = P(A) + P(B) = \frac{1}{3} + \frac{3}{4} = \frac{13}{12} > 1$.\\
		Pero la probabilidad de todo evento tiene que estar entre 0 y 1,
		por lo que la suposición de que A y B eran mutuamente excluyentes debe de ser falsa.\\
		Por lo tanto, A y B no pueden ser mutuamente excluyentes.
	}

	% Ejercicio 5
   \item {
    En cierto pueblecillo con 100,000 habitantes se publican tres periodicos I,II,III. La proporción es la siguiente:\\
I:.10 , II:.30, III:.05, I y II: .08, I y III: .02, II y III: .01 , I,II,III: .01
	\begin{enumerate}[label=\alph*) ]
	% a
   \item {
   Encuentre el número de personas que leen sólo un periódico.\\
	El numero de personas que lean un sólo periódico está dado por la probabilidad:\\
	$P(I \cap II^c\cap III^c)+P(I^c \cap II\cap III^c)+P(I^c \cap II^c\cap III)$ y esto es:\\
	$P(I \cap( II^c \cap III^c))+ P(II\cap(I^c \cap III^c))+P(III \cap (I^c \cap II^c))=$ \\
	$P(I \cap (II \cup III)^c)+P(II \cap (I \cup III)^c)+P(III \cap (II \cup I)^c)=$\\
	$P(I-(II\cup III))+P(II-(I \cup III))+P(III-(II \cup I))=$\\
	$P(I)-P(I \cap (II \cup III))+P(II)-P(II \cap (I \cup III))+P(III)-P(III\cap (II\cup I))=$ \\
	$P(I)+P(II)+P(III)-P((I\cap II)\cup(I \cap III))-P((II\cap I)\cup(II \cap III))-P((III\cap II)\cup(III \cap I))=$\\
	$P(I)+P(II)+P(III)-2P(I\cap II)-2P(I\cap III)-2P(II \cap III)+3P(I\cap II \cap III)$ \\
	Como sabemos cada una de ellas esto es:\\
	$.10+.30+.05-2(.08)-2(.02)-2(.01)+3(.01)=.10+.30+.05-.16-.04-.02+0.3=.26$\\
	Esto quiere decir que el 26 por ciento de la poblacion lee solamente un periodico. Son 100,000 habitantes por lo tanto 26,000 personas leen solamente un periodico.
   }

   % b
   \item {
   ¿Cuántas personas leen al menos dos periódicos?\\
   Si restamos de las 100,000 personas las personas que leen solamente un periodico nos quedamos con las personas que leen mas de un periodico, es decir, que al menos leen dos. Por lo tanto 100,000-26,000 = 74,000 personas leen por lo menos dos periodicos.

   }

      % c
   \item {
  Si I y III son los periodicos matutinos y II es el periódico vespertino, ¿Cuántas perosnas leen al menos un periódico matutino y un vespertino?\\
  Como es al menos un periodico matutino, se puede dar el caso en que se lean los dos. La probabilidad esta dada por:\\
  $P(I\cap III^c \cap II)+P(I^c \cap III \cap II)+P(I\cap II \cap III)$ \\
  $P((I\cap II)-III)+P((II\cap III)-I)+0.01=$ \\
  $P(I\cap II)-P(I\cap II \cap III)+P(II\cap III)-P(II \cap I \cap III)+0.01= 0.08+0.01+0.1-2(0.01)=0.08$ \\ \\
  Entonces hay 8,000 personas que leen al menos un periodico matutino y un vespertino.

   }

	\end{enumerate}

    }

	% Ejercicio 6
   \item {
    Demuestre que:\\
	$$P(A)P(B)-P(A \cap B)=P(A^c \cap B )- P(A^c)P(B)$$ \\
	Por axiomas de probabilidad tenemos que: \\
	\textbf{Definicion:}$P(A^c)=1-P(A)$ $\therefore$ \\
	$-P(A^c)P(B)=-(1-P(A))(P(B))=(P(A)-1)(P(B))=P(A)P(B)-P(B)$\\
	Y además, por teoría de conjuntos: \textbf{Afirmacion} $A^c \cap B = B-A$ por lo que:\\
	$P(A^c \cap B) = P(B-A)$ y:\\
	\textbf{Definicion:}$P(B-A)=P(B)-P(A \cap B)$ $\therefore$\\
	$P(A^C \cap B) - P(A^c)P(B)= P(B)-P(A\cap B)+P(A)P(B)-P(B)=P(A)P(B)-P(A\cap B)_{\blacksquare}$\\
	}

	% Ejercicio 7
   \item {
    Demuestre que:\\
	$$P(A \cup B \cup D) = P(A)+P(A^c \cap B) +P(A^c \cap B^c \cap D)$$\\
	Desarrollando el lado izquierdo tenemos que:\\
	$P(A \cup B \cup D) = P(A \cup (B\cup D))=P(A)+P(B\cup D) - P(A \cap (B\cup D)) = P(A)+P(B)+P(D)-P(B\cap D)-P((A\cap B) \cup (A \cap D))= \\ P(A)+P(B)+P(D)-P(B \cap D) -P(A\cap B) -P(A\cap D) +P(A\cap B \cap D)  \textbf{(1)}$\\\\
	Y desarrollando el lado derecho tenemos que:\\
	Por lo visto en el inciso anterior: $P(A^c \cap B) = P(B-A) = P(B)-P(A\cap B)$ y tenemos que:\\
	$P(A^c \cap B^c \cap D) = P( (A^c \cap B^c) \cap D)$ y por teoria de conjuntos: $(A^c \cap B^c) = (A \cup B)^c$ por lo tanto:\\
	$P((A^c \cap B^c))\cap D)= P((A \cup B)^c \cap D)$ y por teoria de conjuntos tenemos que:\\
	\textbf{Afirmacion:} $(A\cup B)^c \cap D = D-(A\cup B)$ entonces:\\
	$P((A\cup B)^c \cap D) = P(D-(A\cup B))$ y por la definicion del inciso anterior tenemos que:\\
	$=P(D)-P(D\cap(A\cup B)) = P(D)-P((D\cap A) \cup (D \cap B))=P(D)-P(D\cap A) - P(D \cap B) + P((D\cap A) \cap ( D \cap B))= P(D) -P(D \cap A) - P(D \cap B) + P(D \cap B \cap A)$ entonces en el lado derecho:\\
	$P(A) + P(A^c \cap B) +P(A^c \cap B^c \cap D) =$\\
	$P(A)+P(B)-P(A\cap B) +P(D) -P(D\cap A) - P(D\cap B) +P(D\cap B \cap A) =$ \\
	$P(A)+P(B)+P(D)-P(A \cap B) -P(D\cap B) -P(D\cap A) +P(D\cap A \cap B) \textbf{(2)}$ \\\\
	Como la intersección es conmutativa entonces el lado izquierdo y derecho son iguales al ser desarrollados. Por lo tanto la igualdad se cumple. $_{\blacksquare}$
	}

	% Ejercicio 8
   \item {
    Demuestre que:\\
	\begin{center}
	$P(E \cup F \cup G) = P(E) + P(F) + P(G) - P(E^c \cap F \cap G) -P(E \cap F^c \cap G) - P(E \cap F \cap G^c)-2P(E \cap F \cap G)$ \\

	\end{center}
	Por el inciso anterior sabemos que al desarrollar el lado izquierdo tenemos que:\\
	$P(E)+P(F)+P(G)-P(F\cap E)-P(F\cap G) -P( G\cap E)+P(F\cap E \cap G)$\\
	Por teoría de conjuntos sabemos que $(F\cap G \cap E^c) = ((F \cap G) \cap E^c)=(F\cap G)-E$ entonces $P(E^c\cap F \cap G) = P(F\cap G) - E) = P(F\cap G) -P((F\cap G) \cap E)= P(F\cap G)-P(F\cap G \cap E)$. De forma análoga se hace lo mismo con $P(E\cap F^c \cap G) $ y $P(E \cap F \cap G^c)$ y tenemos que:
	$-P(E^c \cap F \cap G) = P(F\cap G \cap E) - P(F\cap G)$\\
	$-P(E \cap F^c \cap G) = P(F\cap G \cap E) - P(E\cap G)$\\
	$-P(E \cap F \cap G^c) = P(F\cap G \cap E) - P(F\cap E)$\\
	Esto implica que:\\
	$P(E)+P(F)+P(G)-P(E^c \cap F \cap G)-P(E \cap F^c \cap G)-P(E \cap F \cap G^c)-2P(E\cap F \cap G)=$\\
	$P(E)+P(F)+P(G)-P(F\cap G)-P(E\cap G)-P(F\cap E) +3P(F\cap G \cap E) -2P(E\cap G \cap F)=$\\
	$P(E)+P(F)+P(G)-P(F\cap G) - P(E \cap G) - P(F \cap E) +P(F\cap G \cap E)$\\
	Y esto es igual al lado izquierdo. Por lo tanto ambos lados son iguales. La igualdad se cumple $_{\blacksquare}$

	}

	% Ejercicio 9
   \item {
    Demuestre que si P y P' son dos medidas de probabilidad definidas sobre el mismo espacio muestral (E), antonces aP+bP' es también una medida de probabilidad, para cualesquiera dos números no negativos a,b que satisfacen a+b=1\\
    \textbf{Definición:} Una medida de probabilidad P es una función que asigna a cada elemento A del $\sigma-$algebra un numero entre 0 y 1. Y además que cumple que:\\
    $0 \leq P(A) \leq 1 : A \in \sigma(a)\subseteq \wp(E)$\\
    $P(E)=1, P(\varnothing)=0$\\
    Si $A_{i} \cap A_{j} = 0 $ \ \ $\forall i \neq j $ entonces\\
    $P(\bigcup\limits_{i=1}^{n} A_{i})= \sum_{i=1}^{n} P(A_{i})$\\
    \textbf{Definicion:}Una $\sigma-$algebra es una clase de subconjuntos del espacio muestral si es cerrada bajo complementos, bajo uniones y si el  vacio se encuentre en el sigma algebra.\\ \\
    Como $a+b=1 : a,b \geq 0$ esto implica que despejando a $a$ y a $b$ tenemos que:\\
    $a=1-b$\\
    $b=1-a$\\
    Pero por hipotesis a,b son numeros no negativos esto implica que:\\
    $1-b \geq 0 \rightarrow 1 \geq b \geq 0$ \ \ \ \ \ \ \ \ \ \
    $1-a \geq 0 \rightarrow 1\geq a \geq 0$ \\
    Entonces a,b son numeros positivos que se encuentra entre 0 y 1. \\
    Sabemos que:\\
    $0 \leq P(x) \leq 1$ \ $\therefore$ $0(a)\leq aP(x) \leq a \therefore $ \\$ 0(a)+bP'(x)\leq aP(x)+bP'(x) \leq a+bP'(x)$ \ \ entonces llegamos a la conclusion de que:\\
    $bP'(x)\leq aP(x)+bP'(x) \leq a+bP'(x)$\\
    Por hipotesis, b es un numero no negativo y tenemos que $0\leq P'(x)\leq 1$ ya que P'(x) es una medida de probabilidad entonces $b(0)\leq bP'(x) \leq b \therefore 0\leq bP'(x)$ entonces:\\
    $0\leq bP'(x)\leq aP(x)+bP'(x)\leq a+bP'(x)$\\
    Supongamos que $a+bP'(x)\geq 1$ esto implica que:\\
    $bP'(x)\geq 1-a$ pero por hipotesis a+b=1 $\therefore$ $1-a=b$ $\therefore bP'(x)\geq b \therefore P'(x)\geq 1 \textbf{!}$\\
    Como P'(x) es una medida de probabilidad entonces P'(x)$\leq 1$, la contradicción vino de suponer que $a+bP'(x)\geq 1$ entonces $a+P'(x)\leq 1$ entonces:\\
    \\
    $0\leq bP'(x)\leq aP(x)+bP'(x)\leq a+bP'(x)\leq 1$ \\
    \\
    $0\leq aP(x)+bP'(x)\leq 1$\\ \\
    Entonces la función (aP+bP')(x)=aP(x)+bP'(x) cumple con la primera condición de que sea una medida de probabilidad. Ahora, tenemos que al evaluar la función (aP+bP') con E tenemos que:\\ \\
    $(aP+bP')(E)=aP(E)+bP'(E)=a(1)+b(1)=1$ \\
    Debido a que por hipotesis a+b=1 y P(E)=P'(E)=1 ya que son medidas de probabilidad. Además, tenemos que al evaluar con $\varnothing$ tenemos:\\ \\
    $(aP+bP')(\varnothing)=aP(\varnothing)+bP'(\varnothing)=a(0)+b(0)=0$\\ \\
    Entonces, la funcion (aP+bP')(x) cumple con la segunda conicion para que sea medida de probabilidad. \\ \\
    Ahora supongamos que $A_{i} \cap A_{j} = 0 $ \ \ $\forall i \neq j $ tenemos que:\\ \\
    $(aP+bP')(\bigcup\limits_{i=1}^{n} A_{i})=aP(\bigcup\limits_{i=1}^{n} A_{i})+bP'(\bigcup\limits_{i=1}^{n} A_{i})=a( \sum_{i=1}^{n} P(A_{i})) + b( \sum_{i=1}^{n} P'(A_{i}))=$ \\
    $\sum_{i=1}^{n} aP(A_{i})+\sum_{i=1}^{n} bP'(A_{i})= \sum_{i=1}^{n} aP(A_{i})+bP'(A_{i})=$ \\ \\$\sum_{i=1}^{n} (aP+bP')(A_{i})$\\ \\
    Como cumple la tercera y ultima condición tenemos que la función (aP+bP')(x) es efectivamente una medida de probabilidad.\\
	}

	% Ejercicio 10
   \item {
    Realice el problema de las cuerdas visto en clase, esta vez con 8 cuerdas.\\

    La probabilidad de que salga un anillo está dada por \\ \\
    $P(\Phi)= \frac{\alpha}{\kappa} : \alpha = $ casos favorables, $\kappa$ = casos totales. \\

    Tenemos que $\kappa$ puede ser contado como $(7*5*3*1)^2$ ya que sin perdida de generalidad tomando las partes de abajo de las cuerdas nos fijamos en alguna de ellas, ésta tiene 7 formas de conectarse con las demás cuerdas abajo, hacemos la unión de las cuerdas y ahora  hay dos cuerdas conectadas, ergo, 6 cuerdas no conectadas, entonces tomamos otra cuerda y ésta tiene 5 cuerdas a las cuales puede conectarse. Ahora quedan 4 cuerdas disponibles sin conectar, tomamos otra cuerda y ésta se puede conectar a 3. Quedan ahora 2 cuerdas disponibles y solo hay una manera de conectarlas. Entonces en la parte de abajo se tienen 7*5*3*1 maneras de conectar cuerdas. Pero se multiplica esa cantidad por si misma pues se hace el mismo procedimiento arriba. Entonces: \\ \\
    $\kappa= (7*5*3)^2$\\

    Ahora, tomando los casos favorables tenemos que sin perdida de generalidad nos situamos en las cuerdas de abajo, cualesquiera cuerda se puede unir con 7 cuerdas. Pero la cuerda de arriba ya no se puede unir con la cuerda con la cual se unió la cuerda de abajo pues se haría un anillo entonces la de arriba se puede conectar a 6 cuerdas, no a 7. Y las cuerdas de abajo no se pueden conectar a esa cuerda a la cual se conectó la cuerda de arriba pues se haría otro anillo entonces las de abajo se pueden conectar solamente a 5, no a 6. Y así se hace sucesivamente. Se tiene entonces que los casos favorables son 7!. \\ \\
    $\alpha = 7! \therefore P(\Phi)=\frac{7!}{(7*5*3*2*1)^2}=\frac{7*6!}{7^2(5*3*1)^2}=\frac{6!}{7(5*3)^2}$
	}


	% Ejercicio 11
   \item {
   	Suponga que un experimento se realiza $n$ veces. Para cualquier evento $E$ del
	espacio muestral considere $v(E)$ el número de veces que el evento $E$ ocurrió.
	sea $g(E)=v(E)/n$. Demuestre que $g$ satisface los axiomas de probabilidad.\\
	\begin{enumerate}[label=\Roman*.]
	\item{
		Se tiene por hipótesis que el experimento se realiza exactamente $n$
		veces, y como por definición, el espacio muestral es el conjunto de
		todos los posibles resultados asociados a un experimento (existen $n$ posibles
		resultados para el experimento), se sigue que
			$$ g(\Omega) = \frac{v(\Omega)}{n} = \frac{n}{n} = 1$$
	}
	\item{
		Dado que un evento puede ocurrir solamente en un número positivo de ocasiones,
		y por hipótesis no puede ocurrir más de $n$ veces (ya que el experimento se
		realiza exactamente $n$ veces, con $n$ posibles resultados) se cumple que:
			$$ n \geq v(E) \geq 0$$
		luego, dividiendo entre $n$, la desigualdad no se afecta, y se tiene:
			$$ \frac{n}{n} \geq \frac{v(E)}{n} \geq \frac{0}{n}$$
			$$ 1 \geq \frac{v(E)}{n} \geq 0$$
		como por definición, se tiene que $g(E) = \frac{v(E)}{n}$, se sigue
			$$ 1 \geq g(E) \geq 0$$
		así, se cumple que $g(E) \geq 0$.
	}
	\item{
		Consideremos $E_1, E_2, E_3, ... \in \mathcal{F}$ un conjunto de eventos
		mutuamente excluyentes (i.e. $E_i \cap E_j = \emptyset$, con $i \neq j$).\\

		Ahora, consideremos :
			$$g\left( \bigcup_{i=1}^{\infty} E_{i} \right)$$
		luego, por definición de $g(E)$, se sigue:
			$$g\left( \bigcup_{i=1}^{\infty} E_{i} \right) =
				\frac{ v\left( \bigcup_{i=1}^{\infty} E_{i} \right)  }{ n }$$
		ahora, como por hipótesis, los eventos $E_1, E_2, E_3, ...$ son mutuamente
		excluyentes (no pueden ocurrir de manera simultánea), se sigue que el
		número de veces que ocurre la unión de todos los eventos, está dada por:
			$$\sum_{i=1}^{\infty} v(E_{i})$$
		con $v(E_i)$ el número de veces que ocurre el evento $E_i$. Así, se tiene:
			$$ g\left( \bigcup_{i=1}^{\infty} E_{i} \right) =
			\frac{ \sum_{i=1}^{\infty} v(E_{i})  }{ n } $$
			$$ = \frac{1}{n} \sum_{i=1}^{\infty} v(E_{i}) $$
			$$ = \sum_{i=1}^{\infty} \frac{1}{n} v(E_{i}) $$
			$$ = \sum_{i=1}^{\infty} \frac{v(E_{i})}{n}  $$
		y finalmente, por la definicion de $g$
			$$ = \sum_{i=1}^{\infty} g(E_i)  $$

		\begin{flushright}
			$_{\square}$
		\end{flushright}
	}
	\end{enumerate}


	}

% Ejercicio 12
   \item {
  	\begin{enumerate}[label=\alph*) ]
	% a
   \item {
	Si P(A) = 0.9 y P(B)=0.8 muestre que $P(A \cap B) \geq 0.7$. En general demuestre la desigualdad de Bonferroni:\\
	$$P(A \cap B) \geq P(A)+P(B) -1$$
   }

   % b
   \item {
 Usa induccion para demostrar la generalización de la desigualdad de Bonferroni para n eventos:\\
 $$P(\bigcap\limits_{i=1}^{n} E_{i}) \geq (\sum_{i=1}^{n} P(E_{i})-(n-1)$$

   }




	\end{enumerate}
	}

	  %Ejercicio 13
  \item{
  Dé un ejemplo de un espacio de probabilidad tal que tres eventos $A_{1},A_{2},A_{3}$ satisfagan que $P(A_{i} \cap A_{j})=P(A_{i})P(A_{j})$ para $i\neq j$ pero que no sean independientes.

  Propongamos el siguiente espacio de probabilidad donde se busca la probabilidad de que aparezca algún elemento del conjunto:\\
  $\Omega = \lbrace a_{1}, a_{2}, a_{3}, a_{4} \rbrace$\\
  $A_{1} = \lbrace a_{1}, a_{2} \rbrace$, $A_{2} = \lbrace a_{2}, a_{3} \rbrace$ y $A_{3} =  \lbrace a_{3}, a_{1} \rbrace$\\
  De ahí, $P(A_{1}) = 0.5$, $P(A_{2}) = 0.5$ y $P(A_{3}) = 0.5$.\\\\
  Tenemos que:\\
  $P(A_{1} \cap A_{2}) = 0.25 = P(A_{1})P(A_{2})$\\
  $P(A_{2} \cap A_{3}) = 0.25 = P(A_{2})P(A_{3})$\\
  $P(A_{1} \cap A_{3}) = 0.25 = P(A_{1})P(A_{3})$\\

  Si los 3 eventos fueran independientes, significaría que: \\$P(A_{1} \cap A_{2} \cap A_{3}) = P(A_{1})P(A_{2})P(A_{3})$\\
  Veamos si es así:\\
  $P(A_{1} \cap A_{2} \cap A_{3}) = 0$\\
  $P(A_{1})P(A_{2})P(A_{3}) = 0.125$\\
  $ 0  \neq 0.125$ \\
  Por lo tanto los tres eventos no son independientes y el ejemplo cumple con la hipótesis.\\
  }


		  %Ejercicio 14
  \item{
 Sea $(\Omega, F, P)$ un espacio de probabilidad. Considere $\lbrace A_{i} \rbrace_{i=1}^{\infty} \subseteq F$. Demuestre que si para todo $i \in N$ se cumple que $P(A_{i})=1$ entonces $P((\bigcap\limits_{i=1}^{\infty}A_{i})=1$

Utilizando la propiedad de complemento y la desigualdad de Boole, tenemos que:
\begin{align*}
P((\bigcap\limits_{i=1}^{\infty}A_{i}) &= 1 - (P(\bigcap\limits_{i=1}^{\infty}A_{i})^c) \\
&= 1- P(\bigcup\limits_{i=1}^{\infty}A^c_{i}) \\
&\geqslant 1- \sum_{i=1}^{\infty} P(A^c_{i}) \\
&\geqslant 1- \sum_{i=1}^{\infty} (1- P(A_{i})) \\
&\geqslant 1- \sum_{i=1}^{\infty} (1- 1) \\
&\geqslant 1- \sum_{i=1}^{\infty} (0) \\
&= 1 _{\blacksquare}
\end{align*}
Ya que todas las $P(A_{i})$ son iguales a 1, tenemos que la desigualdad está acotada por arriba y por abajo para darnos de resultado final 1.



  }

  		  %Ejercicio 15
  \item{
 Demuestre que la intersección de dos $\sigma$-algebras es $\sigma$-algebras.\\

Sea $\mathcal{F}_{1}$ y $\mathcal{F}_{2}$ dos $\sigma$-álgebras de subconjuntos de $\Omega$.
Entonces $\mathcal{F}_{1} \cap \mathcal{F}_{2}$ es la colección de subconjuntos de $\Omega$ cuyos elementos pertenecen tanto a $\mathcal{F}_{1}$ como a $\mathcal{F}_{2}$.
Ahora demostremos que $\mathcal{F}_{1} \cap \mathcal{F}_{2}$ cumple las 3 propiedades de una $\sigma$-álgebra:\\
 	Como  $\mathcal{F}_{1}$ y $\mathcal{F}_{2}$ son $\sigma$-álgebras, entonces $\Omega 	\in \mathcal{F}_{1}$ y $\Omega \in \mathcal{F}_{1}$. Por lo tanto, $\Omega \in\mathcal{F}_{1} \cap \mathcal{F}_{2}$ ...(1)\\
 	Sea A un elemento en $\mathcal{F}_{1} \cap \mathcal{F}_{2}$. Entonces $A \in \mathcal{F}_{1}$ y $A \in \mathcal{F}_{2}$. Por lo tanto, $A^c \in \mathcal{F}_{1}$ y $A^c \in \mathcal{F}_{2}$, es decir, $A^c \in\mathcal{F}_{1} \cap \mathcal{F}_{2}$ ... (2)\\
 	Sea $A_{1},A_{2},...$ una sucesión de elementos en $\mathcal{F}_{1} \cap \mathcal{F}_{2}$. Entonces $A_{1},A_{2},... \in \mathcal{F}_{1}$ y $A_{1},A_{2},... \in \mathcal{F}_{2}$. Por lo tanto, $\bigcup\limits_{i=1}^{\infty}A_{i}  \in \mathcal{F}_{1}$ y  $\bigcup\limits_{i=1}^{\infty}A_{i}  \in \mathcal{F}_{2}$. De donde tenemos que, $\bigcup\limits_{i=1}^{\infty}A_{i}  \in \mathcal{F}_{1} \cap \mathcal{F}_{2}$...(3)\\
 	\\De (1), (2) y (3) tenemos que $\mathcal{F}_{1} \cap \mathcal{F}_{2}$ es una  $\sigma$-álgebra. $_{\blacksquare}$\\

  }


  	  %Ejercicio 16
  \item{
Si 4 matrimonios son acomodados aleatoriamente en una fila ¿cual es la probabilidad de que ningún hombre quede sentado junto a su esposa?\\

Dejemos $E_{i}, i= 1,2,3,4$ como el evento donde el iésimo matrimonio se sienta junto, se sigue que la probabilidad deseada es $1 - P(\bigcup\limits_{i=1}^{4}E_{i})$. Ahora, de la ley de inclusión-exclusión tenemos que:
\\$P(\bigcup\limits_{i=1}^{4}E_{i}) = \sum_{i=1}^{4} P(E_{i}) - ... + (-1)^{n-1}\sum_{i_{1} < i_{2} < ... < i_{n}}P(E_{i_{1}}E_{i_{2}}...E_{i_{n}}) + ... - P(E_{1}E_{2}...E_{4})$.\\
\\Para calcular $E_{i_{1}}E_{i_{2}}...E_{i_{n}}$, primero tenemos que notar que hay 8! maneras de acomodar a 8 personas en una fila. El número de arreglos que resulta en un conjunto específico de $n$ hombres sentados con sus esposas puede ser obtenido de pensar en cada uno de los $n$ matrimonios como entidades solas. Entonces tenemos $(8-n)!$ arreglos. Finalmente, como cada uno de los $n$ matrimonios puede acomodarse uno junto al otro en una de dos posibles maneras, se sigue que hay $2^n(8-n)!$ arreglos que resultan en un conjunto específico de $n$ hombres que se sientan junto a sus esposas. Por lo tanto,\\
\\$P(E_{i_{1}}E_{i_{2}}...E_{i_{n}}) = \frac{2^n(8-n)!}{(8!)}$.\\
\\Entonces, de la ley de inclusión-exclusión obtenemos que la probabilidad de que al menos un matrimonio se siente junto es de:\\
\\$ {4 \choose 1}\frac{(2^1)(7!)}{(8!)} - {4 \choose 2}\frac{(2^2)(6!)}{(8!)} + {4 \choose 3}\frac{(2^3)(5!)}{(8!)} - {4 \choose 4}\frac{(2^4)(4!)}{(8!)} \approx 0.6571$\\
\\La probabilidad deseada es de aproximadamente 0.3428.

  }

  	  %Ejercicio 17
  \item{
	Un closet contiene 10 pares de zapatos. Si 8 de ellos se seleccionan al
	azar ¿cual es la probabilidad de que no se complete ningun par? ¿Cual es la
	probabilidad de que se complete exactamente un par?

	\begin{enumerate}
	\item{
	Sea $P(A)$ la probabilidad de que no se complete ningún par.\\
	Como no debe completarse ningún par al tomar los ocho zapatos, debemos
	elegir a lo más un zapato de cada posible par. Así, elegimos ocho pares de
	los diez disponibles, esto está dado mediante:
		$${10}\choose{8}$$
	Luego, de los ocho pares, contamos las formas en las que podemos elegir un
	zapato de cada uno de ellos, esto es:
		$$2^8$$
	puesto que por cada par podemos elegir el zapato izquierdo o el derecho.
	Así, los casos favorables entre el número total de formas de elegir
	ocho zapatos de entre veinte ${20}\choose{8}$, nos dan la probabilidad del
	evento, a saber:
		$$P(A) = \frac{ {10 \choose 8} 2^8 }{ {20 \choose 8} } \approx 0.0914 $$
	}

	\item{
	Sea $P(A)$ la probabilidad de que se complete exactamente un par.\\
	Para esto, fijamos un par de zapatos, (el cual es el único que se va a
	completar), así, tenemos
		$${10 \choose 1} = 10$$
	formas de escoger exactamente un par.
	Luego, para los seis zapatos restantes, fijamos un zapato por par, de forma
	que no se complete otro par, es decir, escogemos seis zapatos de entre los
	nueve pares restantes, esto está dado por
		$${9}\choose{6}$$
	y podemos elegir un zapato de cada par de $2^6$ maneras distintas, así, la
	probabilidad del evento está dada por:
		$$P(A) = \frac{ {9 \choose 6} 2^6 * 10}{ {20 \choose 8} } \approx 0.4267 $$
	}
	\end{enumerate}
  }

    %Ejercicio 18
  \item{
	Una urna contiene $N$ bolas numeradas de $1$ a $N$. Las primeras $N_{1}$ son
	defectuosas y las restantes $N_{2}$ no son defectuosas. Se seleccionan $n$
	bolas de la urna. Sea $A$ la muestra de $n$ bolas contiene exactamente
	$n_{1}$ bolas defectuosas.
	Calcula $P(A)$ si las bolas se seleccionan con reemplazo.\\

	Sea $P(A)$ la probabilidad de que una muestra de $n$ bolas contenga exactamente
	$n_1$ bolas defectuosas, seleccionado las bolas con reemplazo.\\

	Se tiene que el número de casos totales está dado mediante:
	 	$$N^n$$
	puesto que se puede elegir cada bola de $N$ maneras distintas.

	Ahora, para contar los casos favorables, tenemos que elegir exactamente $n_1$
	bolas defectuosas, podemos elegir las $n_1$ bolas de entre la muestra de tamaño
	$n$ mediante:
		$${n}\choose{n_1}$$
	y luego, multiplicamos las formas de elegir cada una de las $n_1$ bolas de entre
	$N_1$ bolas defectuoss
		$$N_1 \cdot N_1 \cdot N_1 ... \cdot N_1 \quad \text{($n_1$ veces)}  = {N_{1}}^{n_1}$$
	puesto que cada una de las $n_1$ bolas puede elegirse de $N_1$ formas distintas

	Análogamante, para elegir las bolas restantes, se sigue que está dado mediante:
		$$(N - N_1) \cdot (N - N_1) \cdot (N - N_1) ... \cdot (N - N_1)
		  \quad \text{($n - n_1$ veces)}  =(N - N_{1})^{n - n_1}$$
	pues no podemos elegir más de $n_1$ bolas defectuosas.\\

	Se sigue por lo tanto que $P(A)$ está dada por:
		$$P(A) = \frac{{n \choose n_1} {N_{1}}^{n_1} (N - N_{1})^{n - n_1} }
				{N^n}$$
  }

     %Ejercicio 19
  \item{
	Supongamos que hay 12 estudiantes en un salón. ¿Cual es la probabilidad de
	que dos de ellos no celebren su cumpleaños en el mismo mes?\\

	Sea $P(A)$ la probabilidad de que dos personas no celeben su cumpleaños en el
	mismo mes.\\

	Se tiene que cada persona puede celebrar su cumpleaños en uno de entre doce
	meses, así, el número de casos totales para 12 personas está dado por
		$$12^{12}$$
	luego, la primera persona puede cumplir su cumpleaños de doce maneras distintas
	(una por cada mes), la segunda de once maneras distintas (puesto que no puede
	cumplir años el mismo mes que la primera), y así sucesivamente, de tal forma
	que:
		$$P(A) = \frac{12 * 11 * 10 * .... * 1}{12^{12}} = \frac{12!}{12^{12}}
				\approx 0.000053723$$
  }





\end{enumerate}
\end{document}
