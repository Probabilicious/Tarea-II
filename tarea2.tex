% ====== TAREA 1 DE PROBABILIDAD ======

\documentclass[12pt,a4paper]{report}
\usepackage[utf8x]{inputenc}
\usepackage{amsmath}
\usepackage{amsfonts}
\usepackage{amssymb}
\usepackage{graphicx}
\usepackage{enumitem}

\newcommand*{\Comb}[2]{{}^{#1}C_{#2}}

\begin{document}
\begin{titlepage}
	\centering
	{\scshape\LARGE Universidad Autónoma de México \par}
	\vspace{1cm}
	{\scshape\Large Probabilidad I\par}
	\vspace{1.5cm}
	{\huge\bfseries Tarea I\par}
	\vspace{.5cm}
	{\Large\itshape Sandra Del Mar Soto Corderi \par}
	\vspace{.5cm}
	{\Large\itshape Edgar Quiroz Castañeda \par}
    \vspace{.5cm}
	{\Large\itshape Raúl Llamosas Alvarado \par}
	 \vspace{.5cm}
	{\Large\itshape Alan Ernesto Arteaga Vázquez \par}
	\vfill
	 \includegraphics[width=0.5\textwidth]{escudo.png}
	\vfill

% Bottom of the page
	{\large Martes 21 de Agosto del 2018 \par}
\end{titlepage}

\pagebreak
\setlength{\voffset}{-0.75in}
\setlength{\headsep}{5pt}

\begin{enumerate}
   % Ejercicio 1
   \item {
  La diferencia simétrica de dos conjuntos A,B se define como:\\

		$$A\triangle B = (A \cap B^c)\cup (B \cap A^c)=(A \cup B)\ (B \cap A)$$
Demuestre
\begin{enumerate}[label=\alph*) ]
	%a
	\item{$A^c \triangle B^c = A \triangle B $ \\
		
	} 
	
	%b
	\item{$ (A_{1} \cup A_{2})\triangle (B_{1} \cup B_{2})\subseteq (A_{1} \triangle B_{1}) \cup (A_{2} \triangle B_{2}) $ \\
		
	} 
	
	%c
	\item{Sea $\Omega = \lbrace 1,2,3,4,5 \rbrace  $ un conjunto y $\tau \subseteq P(\Omega)$ una $\sigma$-algebra. Demuestre que $A \triangle B \in \tau $ siempre que $A,B \in \tau $ \\
		
	} 
	
	
	
\end{enumerate}

	}

	% Ejercicio 2
   \item {
    Si $\Omega = \lbrace 1,2,3,4,5 \rbrace$ construya una $\sigma$-algebra $F \neq P(\Omega)$ tal que:\\
    $$\lbrace \lbrace 1,2 \rbrace , \lbrace 1,3 \rbrace \rbrace \subset F$$ 
		
	}


	% Ejercicio 3
   \item {
   En una universidad se ofrecen tres cursos de idiomas: uno  de inglés, uno de francés y otro de alemán. Cualquiera de los 100 alumnos de la escuela puede tomar dichos cursos. Hay 26 estudiantes en la clase de inglés, 28 en la de francés y 16 en la de alemán. Hay 12 estudiantes que están en el de inglés y francés, 4 en el de ingles y alemán , 6 en el de alemán y francés y 2 tomando los tres cursos\\

	\begin{enumerate}[label=\alph*) ]
	% a)
   \item {
	Si un estudiante se selecciona aleatoriamente ¿Cual es la probabilidad de que no esté inscrito en ningún curso de idiomas?\\


   }

   % b)
   \item {
   Si un estudiante se selecciona aleatoriamente, ¿Cual es la probabilidad de que lleve exactamente uno de los cursos de idiomas?\\

   }
   
    % c)
   \item {
  Si dos estudiantes se seleccionan aleatoriamente ¿Cual es la probabilidad de que al menos uno esté tomando alguna clase de idiomas?\\

   }
   
   
	\end{enumerate}

	}

	% Ejercicio 4
   \item {
  	Si $P(A) = \frac{1}{3}$ y $P(B^c)=\frac{1}{4}$. ¿Pueden ser A y B mutuamente excluyentes? 
	}

	% Ejercicio 5
   \item {
    En cierto pueblecillo con 100,000 habitantes se publican tres periodicos I,II,III. La proporción es la siguiente:\\
I:10 , II:30, III:5, I y II: 8, I y III: 2, II y III: 1 , I,II,III: 1
	\begin{enumerate}[label=\alph*) ]
	% a
   \item {
   Encuentre el número de personas que leen sólo un periódico.\\

   }

   % b
   \item {
   ¿Cuántas personas leen al menos dos periódicos?\\

   }
   
      % c
   \item {
  Si I y III son los periodicos matutinos y II es el periódico vespertino, ¿Cuántas perosnas leen al menos un periódico matutino y un vespertino?\\

   }

	\end{enumerate}

    }

	% Ejercicio 6
   \item {
    Demuestre que:\\
	$$P(A)P(B)-P(A \cap B)=P(A^c \cap B )- P(A^c)P(B)$$
	}
	
	% Ejercicio 7
   \item {
    Demuestre que:\\
	$$P(A \cup B \cup D) = P(A)+P(A^c \cap B) +P(A^c \cap B^c \cap D)$$
	}
	
	% Ejercicio 8
   \item {
    Demuestre que:\\
	\begin{center}
	$P(E \cup F \cup G) = P(E) + P(F) + P(G) - P(E^c \cap F \cap G) -P(E \cap F^c \cap G) - P(E \cap F \cap G^c)-2P(E \cap F \cap G)$
	\end{center}
	}
	
	% Ejercicio 9
   \item {
    Demuestre que si P y P' son dos medidas de probabilidad definidas sobre el mismo espacio muestral, antonces aP+bP' es también una medida de probabilidad, para cualesquiera dos números no negativos a,b que satisfacen a+b=1\\
	}
	
	% Ejercicio 10
   \item {
    Realice el problema de las cuerdas visto en clase, esta vez con 8 cuerdas.\\
	}
	
		
	% Ejercicio 11
   \item {
   	Suponga que un experimento se realiza n veces. Para cualquier evento E del espacio muestral considere v(E) el número de veces que el evento E ocurrió. sea g(E)=v(E)/n. Demuestre que g satisface los axiomas de probabilidad.\\
	}
	
% Ejercicio 12
   \item {
  	\begin{enumerate}[label=\alph*) ]
	% a
   \item {
	Si P(A) = 0.9 y P(B)=0.8 muestre que $P(A \cap B) \geq 0.7$. En general demuestre la desigualdad de Bonferroni:\\
	$$P(A \cap B) \geq P(A)+P(B) -1$$
   }

   % b
   \item {
 Usa induccion para demostrar la generalización de la desigualdad de Bonferroni para n eventos:\\
 $$P(\bigcap\limits_{i=1}^{n} E_{i}) \geq (\sum_{i=1}^{n} P(E_{i})-(n-1)$$

   }
   

  

	\end{enumerate}
	}
	
	  %Ejercicio 13
  \item{
  Dé un ejemplo de un espacio de probabilidad tal que tres eventos $A_{1},A_{2},A_{3}$ satisfagan que $P(A_{i} \cap A_{j})=P(A_{i})P(A_{j})$ para $i\neq j$ pero que no sean independientes.
  }
  
	
		  %Ejercicio 14
  \item{
 Sea $(\Omega, F, P)$ un espacio de probabilidad. Considere $\lbrace A_{i} \rbrace_{i=1}^{\infty} \subseteq F$. Demuestre que si para todo $i \in N$ se cumple que $P(A_{i})=1$ entonces $P((\bigcap\limits_{i=1}^{\infty}A_{i})=1$
  }
  
  		  %Ejercicio 15
  \item{
 Demuestre que la intersección de dos $\sigma$-algebras es $\sigma$-algebras.
  }
  
  
  	  %Ejercicio 16
  \item{
Si 4 matrimonios son acomodados aleatoriamente en una fila ¿cual es la probabilidad de que (ella me ame) ningún hombre quede sentado junto a su esposa?
  }
  
  	  %Ejercicio 17
  \item{
Un closet contiene 10 pares de zapatos. Si 8 de ellos se seleccionan al azar ¿cual es la probabilidad de que no se complete ningun par? ¿Cual es la probabilidad de que se complete exactamente un par? 
  }
  
    %Ejercicio 18
  \item{
	Una urna contiene N  bolas numeradas de 1 a N. Las primeras $N_{1}$ son defectosas y las restantes $N_{2}$ no son defectuosas. Se seleccionan n bolas de la urna. Sea A la muestra de n bolas contiene $n_{1}$ bolas defectuosas. calcula P(A) si las bolas se seleccionan con reemplazo. 
  }
  
     %Ejercicio 19
  \item{
	Supongamos que hay 12 estudiantes en un salón. ¿Cual es la probabilidad de que dos de ellos no celebren su cumpleaños en el mismo mes? 
  }
  
  
  
  
	
\end{enumerate}
\end{document}
